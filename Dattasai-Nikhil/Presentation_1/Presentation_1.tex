%%%%%%%%%%%%%%%%%%%%%%%%%%%%%%%%%%%%%%%%%
% Beamer Presentation
% LaTeX Template
% Version 1.0 (10/11/12)
%
% This template has been downloaded from:
% http://www.LaTeXTemplates.com
%
% License:
% CC BY-NC-SA 3.0 (http://creativecommons.org/licenses/by-nc-sa/3.0/)
%
%%%%%%%%%%%%%%%%%%%%%%%%%%%%%%%%%%%%%%%%%

%----------------------------------------------------------------------------------------
%	PACKAGES AND THEMES
%----------------------------------------------------------------------------------------

\documentclass{beamer}

\mode<presentation> {

% The Beamer class comes with a number of default slide themes
% which change the colors and layouts of slides. Below this is a list
% of all the themes, uncomment each in turn to see what they look like.

%\usetheme{default}
%\usetheme{AnnArbor}
%\usetheme{Antibes}
%\usetheme{Bergen}
%\usetheme{Berkeley}
%\usetheme{Berlin}
%\usetheme{Boadilla}
%\usetheme{CambridgeUS}
%\usetheme{Copenhagen}
%\usetheme{Darmstadt}
%\usetheme{Dresden}
%\usetheme{Frankfurt}
%\usetheme{Goettingen}
%\usetheme{Hannover}
%\usetheme{Ilmenau}
%\usetheme{JuanLesPins}
%\usetheme{Luebeck}
\usetheme{Madrid}
%\usetheme{Malmoe}
%\usetheme{Marburg}
%\usetheme{Montpellier}
%\usetheme{PaloAlto}
%\usetheme{Pittsburgh}
%\usetheme{Rochester}
%\usetheme{Singapore}
%\usetheme{Szeged}
%\usetheme{Warsaw}

% As well as themes, the Beamer class has a number of color themes
% for any slide theme. Uncomment each of these in turn to see how it
% changes the colors of your current slide theme.

%\usecolortheme{albatross}
%\usecolortheme{beaver}
%\usecolortheme{beetle}
%\usecolortheme{crane}
%\usecolortheme{dolphin}
%\usecolortheme{dove}
%\usecolortheme{fly}
%\usecolortheme{lily}
%\usecolortheme{orchid}
%\usecolortheme{rose}
%\usecolortheme{seagull}
%\usecolortheme{seahorse}
%\usecolortheme{whale}
%\usecolortheme{wolverine}

%\setbeamertemplate{footline} % To remove the footer line in all slides uncomment this line
%\setbeamertemplate{footline}[page number] % To replace the footer line in all slides with a simple slide count uncomment this line

%\setbeamertemplate{navigation symbols}{} % To remove the navigation symbols from the bottom of all slides uncomment this line
}

\usepackage{graphicx} % Allows including images
\usepackage{booktabs} % Allows the use of \toprule, \midrule and \bottomrule in tables

%----------------------------------------------------------------------------------------
%	TITLE PAGE
%----------------------------------------------------------------------------------------

\title[Short title]{Optimization } % The short title appears at the bottom of every slide, the full title is only on the title page

\author{MA17BTECH11002,MA17BTECH11007} % Your name
\institute[UCLA] % Your institution as it will appear on the bottom of every slide, may be shorthand to save space
{
Convexity of functions. \\ % Your institution for the title page
\medskip
\textit{Problem 1.5} % Your email address
}
\date{February 20, 2019} % Date, can be changed to a custom date

\begin{document}

\begin{frame}
\titlepage % Print the title page as the first slide
\end{frame}
%convex function slide.
\begin{frame}
\frametitle{Convex functions} % Table of contents slide, comment this block out to remove it
 % Throughout your presentation, if you choose to use \section{} and \subsection{} commands, these will automatically be printed on this slide as an overview of your presentation
\begin{block}{Definition}
A real-valued function defined on an n-dimensional interval is said to be convex if the line segment between any two points on the graph of the function lies above or on the graph.
\end{block}

A single variable function f is said to be convex if:\newline

\quad \quad \quad \quad \quad  $f[$\lambda$x+(1-\lambda)y] \quad \leq  \quad $\lambda$f(x)+(1-\lambda)f(y)$\newline

\quad \quad \quad \quad \quad \quad\quad \quad \quad \quad \quad \quadfor for 0$<$$\lambda$$<$1\newline

Examples: $x^{2}$,$x^{3}$ in $R^{+}$,$e^{x}$ etc.

%----------------------------------------------------------------------------------------
%	PRESENTATION SLIDES
%----------------------------------------------------------------------------------------

%------------------------------------------------

\end{frame}
%concave function slide.
\begin{frame}
\frametitle{Concave functions} % Table of contents slide, comment this block out to remove it
 % Throughout your presentation, if you choose to use \section{} and \subsection{} commands, these will automatically be printed on this slide as an overview of your presentation
\begin{block}{Definition}
A real-valued function defined on an n-dimensional interval is said to be concave if the line segment between any two points on the graph of the function lies below or on the graph.
\end{block}

A single variable function f is said to be concave if:\newline

\quad \quad \quad \quad \quad  $f[$\lambda$x+(1-\lambda)y] \quad \geq  \quad $\lambda$f(x)+(1-\lambda)f(y)$\newline

\quad \quad \quad \quad \quad \quad\quad \quad \quad \quad \quad \quadfor for 0$<$$\lambda$$<$1\newline

Examples: $x^{3}$ in $R^{-}$,lnx etc.

%----------------------------------------------------------------------------------------
%	PRESENTATION SLIDES
%----------------------------------------------------------------------------------------

%------------------------------------------------

\end{frame}
%problem statement.
\begin{frame}
\frametitle{Problem}
Let f(z) =xy , z$\in$$R^{2}$\newline

Sketch f(z) and deduce wheather it is convex.Theoritically explain your observation.

\end{frame}

%solution.

%------------------------------------------------

\begin{frame}[fragile] % Need to use the fragile option when verbatim is used in the slide
\frametitle{Python Code}
\begin{example}
\begin{verbatim}
    import numpy as np
    import matplotlib.pylab as plt
    from mpl_toolkits.mplot3d import Axes3D
    def fnc(X):
        return (X[0]* X[1])
    fig = plt.figure()
    ax = fig.add_subplot(111, projection=Axes3D.name)
    x = y = np.linspace(-50,50,100)
   
\end{verbatim}
\end{example}
\end{frame}


\begin{frame}[fragile] % Need to use the fragile option when verbatim is used in the slide
\frametitle{Python Code}
\begin{example}
\begin{verbatim} 
    X, Y = np.meshgrid(x, y)
    Z = fnc([X,Y])
    ax.plot_surface(X, Y, Z)
    ax.set_xlabel('X Label')
    ax.set_ylabel('Y Label')
    ax.set_zlabel('Z Label')
    ax.view_init(elev=15, azim=-118)
    plt.show()
\end{verbatim}
\end{example}
\end{frame}

%------------------------------------------------


\begin{frame}
\frametitle{Solution}

The function f(z) is neither convex nor concave as a whole because it behaves as a convex function in certain regions and concave in certain regions.\newline

If we look at the two quadrants where $x>0$ , $y>0$ and $x<0$ , $y<0$ we can clearly observe by the below figure that at some points a straight line drawn between two points is always above the graph.\newline

Similarly,if we look at the two quadrants where $x>0$ , $y<0$ and $x<0$ , $y>0$ we can clearly observe by the below figure that at some points a straight line drawn between two points is always below the graph.



\end{frame}


%graphs of the functions.
\begin{frame}

\begin{figure}
\includegraphics[width=0.8\linewidth]{xyconvex.JPG}
\end{figure}
    
\end{frame}
%graphs of the functions.
\begin{frame}

\begin{figure}
\includegraphics[width=0.8\linewidth]{xyconcave.JPG}
\end{figure}
    
\end{frame}

%------------------------------------------------
\begin{frame}

The notion of mathematical definition for single variable can be extended to n-dimentional variables where x=($x_{1}$,$x_{2}$,$x_{3}$,....$x_{n}$);y=($y_{1}$,$y_{2}$,$y_{3}$,....$y_{n}$)\newline

from the definition, consider $f[$\lambda$x+(1-\lambda)y]-[ $\lambda$f(x)+(1-\lambda)f(y)$]\newline

consider two points $(x_{1},y_{1})$and$(x_{2},y_{2})$.\newline.

calculating $f[$\lambda(x_{1},y_{1})$+(1-\lambda)(x_{2},y_{2})]-[ $\lambda$f(x_{1},y_{1})+(1-\lambda)f(x_{2},y_{2})$]\newline

$\Rightarrow$  $f[$\lambda$x_{1}$+(1-\lambda)x_{2}$,$\lambda$y_{1}$+(1-\lambda)y_{2}$]-[ $\lambda$f(x_{1},y_{1})+(1-\lambda)f(x_{2},y_{2})$]\newline

$\Rightarrow$ $\lambda^{2}$$x_{1}y_{1}$+$(1-\lambda)^{2}$$x_{2}y_{2}$+$\lambda(1-\lambda)$$(x_{1}y_{2}+x_{2}y_{1})$-[$\lambda$$x_{1}y_{1}$+$(1-\lambda)$$x_{2}y_{2}$]\newline

$\Rightarrow$$(\lambda^{2}-\lambda)(x_{1}y_{1}+x_{2}y_{2})$+$(\lambda-\lambda^{2})(x_{1}y_{2}+x_{2}y_{1})$\newline

$\Rightarrow$$(\lambda^{2}-\lambda)(x_{1}y_{1}+x_{2}y_{2}-x_{1}y_{2}-x_{2}y_{1})$\newline

$\Rightarrow$$(\lambda^{2}-\lambda)[x_{1}(y_{1}-y_{2})+x_{2}(y_{2}-y_{1})]$\newline

\end{frame}
\begin{frame}
    


$\Rightarrow$$(\lambda^{2}-\lambda)(x_{1}-x_{2})(y_{1}-y_{2})$\newline
for those  points where $(x_{1}-x_{2})(y_{1}-y_{2})\leq0$ it takes a negative value, hence it shows that at those points \newline

Hence,from the above equation we can conclude that for those  points where $(x_{1}-x_{2})(y_{1}-y_{2})\geq0$ it takes a negative value, hence it shows that at those points \newline
$f[$\lambda$x+(1-\lambda)y] \leq $\lambda$f(x)+(1-\lambda)f(y)$\newline

Which concludes that it is convex at those points.\newline

For those  points where $(x_{1}-x_{2})(y_{1}-y_{2})\leq0$ it takes a positive value, hence it shows that at those points \newline
$f[$\lambda$x+(1-\lambda)y] \geq $\lambda$f(x)+(1-\lambda)f(y)$\newline

Which concludes that it is concave at those points.

\end{frame}


\begin{frame}
    
Final Conclusion: \newline \newline
      The function is convex where: \newline
         $x_{1}>x_{2} , y_{1}>y_{2}$ and
         $x_{1}<x_{2} , y_{1}<y_{2}$ \newline
      The function is convex where: \newline
         $(x_{1}<x_{2}) , (y_{1}>y_{2})$ and
         $(x_{1}>x_{2}) , (y_{1}<y_{2})$
\end{frame}

%-----------------------



\begin{frame}
\Huge{\centerline{The End}}
\quad \quad\quad \quad \quad Thank You!
\end{frame}

%----------------------------------------------------------------------------------------

\end{document}