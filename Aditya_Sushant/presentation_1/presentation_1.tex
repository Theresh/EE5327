%%%%%%%%%%%%%%%%%%%%%%%%%%%%%%%%%%%%%%%%%
% Beamer Presentation
% LaTeX Template
% Version 1.0 (10/11/12)
%
% This template has been downloaded from:
% http://www.LaTeXTemplates.com
%
% License:
% CC BY-NC-SA 3.0 (http://creativecommons.org/licenses/by-nc-sa/3.0/)
%
%%%%%%%%%%%%%%%%%%%%%%%%%%%%%%%%%%%%%%%%%

%----------------------------------------------------------------------------------------
%	PACKAGES AND THEMES
%----------------------------------------------------------------------------------------

\documentclass{beamer}

\mode<presentation> {

% The Beamer class comes with a number of default slide themes
% which change the colors and layouts of slides. Below this is a list
% of all the themes, uncomment each in turn to see what they look like.

%\usetheme{default}
%\usetheme{AnnArbor}
%\usetheme{Antibes}
%\usetheme{Bergen}
%\usetheme{Berkeley}
%\usetheme{Berlin}
%\usetheme{Boadilla}
%\usetheme{CambridgeUS}
%\usetheme{Copenhagen}
%\usetheme{Darmstadt}
%\usetheme{Dresden}
%\usetheme{Frankfurt}
%\usetheme{Goettingen}
%\usetheme{Hannover}
%\usetheme{Ilmenau}
%\usetheme{JuanLesPins}
%\usetheme{Luebeck}
\usetheme{Madrid}
%\usetheme{Malmoe}
%\usetheme{Marburg}
%\usetheme{Montpellier}
%\usetheme{PaloAlto}
%\usetheme{Pittsburgh}
%\usetheme{Rochester}
%\usetheme{Singapore}
%\usetheme{Szeged}
%\usetheme{Warsaw}

% As well as themes, the Beamer class has a number of color themes
% for any slide theme. Uncomment each of these in turn to see how it
% changes the colors of your current slide theme.

%\usecolortheme{albatross}
%\usecolortheme{beaver}
%\usecolortheme{beetle}
%\usecolortheme{crane}
%\usecolortheme{dolphin}
%\usecolortheme{dove}
%\usecolortheme{fly}
%\usecolortheme{lily}
%\usecolortheme{orchid}
%\usecolortheme{rose}
%\usecolortheme{seagull}
%\usecolortheme{seahorse}
%\usecolortheme{whale}
%\usecolortheme{wolverine}

%\setbeamertemplate{footline} % To remove the footer line in all slides uncomment this line
%\setbeamertemplate{footline}[page number] % To replace the footer line in all slides with a simple slide count uncomment this line

%\setbeamertemplate{navigation symbols}{} % To remove the navigation symbols from the bottom of all slides uncomment this line
}

\usepackage{graphicx} % Allows including images
\usepackage{booktabs} % Allows the use of \toprule, \midrule and \bottomrule in tables

%----------------------------------------------------------------------------------------
%	TITLE PAGE
%----------------------------------------------------------------------------------------

\title[3.8-3.11]{Solutions to  Problems 3.8-3.11} % The short title appears at the bottom of every slide, the full title is only on the title page

\author{Adithya Hosapate \\ Sushant Meena} % Your name
\institute[IITH] % Your institution as it will appear on the bottom of every slide, may be shorthand to save space
{
IIT Hyderabad \\ % Your institution for the title page
\medskip
\textit{ee16btech11040@iith.ac.in} % Your email address 
\medskip \\
\textit{es16btech11021@iith.ac.in}
}
\date{\today} % Date, can be changed to a custom date

\begin{document}

\begin{frame}
\titlepage % Print the title page as the first slide
\end{frame}



%----------------------------------------------------------------------------------------
%	PRESENTATION SLIDES
%----------------------------------------------------------------------------------------

%------------------------------------------------


\begin{frame}
\frametitle{Question 3.8}
Find graphically,
\begin{equation*}
    \min_{x} f(\textbf{x}) =(x_1-8)^2+(x_2-6)^2 
\end{equation*}
Subject to the constraint $g(\textbf{x})=x_1+x_2-18=0$\\
Solution:\\
\begin{center}
   \includegraphics[scale=0.3]{kapp.png}
 
\end{center}

\end{frame}




%------------------------------------------------

\begin{frame}
\frametitle{Question 3.9}
Solve Using Lagrangian,
\begin{equation*}
    \min_{x} f(\textbf{x}) =(x_1-8)^2+(x_2-6)^2 
\end{equation*}
Subject to the constraint $ g( \textbf{x} )=x_1+x_2-18=0 $\\
Solution:\\
\begin{equation*}
   L(\textbf{x},\lambda)=(x_1-8)^2+(x_2-6)^2 + \lambda(x_1+x_2-18)
\end{equation*}

By Stationarity,

\begin{equation*}
   \nabla_{x} L( \textbf{x}^{opt} , \lambda )=0
\end{equation*}
This gives the 2 equations $2x_1 - 16 - \lambda =0$,$2x_2 - 12 - \lambda=0$.





\end{frame}


%------------------------------------------------



%------------------------------------------------

\begin{frame}
\frametitle{Question 3.9}

 2 equations $2x_1 - 16 - \lambda =0$, $2x_2 - 12 - \lambda=0$.\\
From primal feasibility, we get the third equation: $x_1+x_2-18=0$.
\begin{equation*}
    \begin{bmatrix}
    2       & 0 & -1  \\
   0       & 2 & -1  \\

   1      & 1 & 0 \\
\end{bmatrix}
  \begin{bmatrix}
    x_1  \\
   x_2  \\

   \lambda \\
\end{bmatrix}
=\begin{bmatrix}
    16  \\
   12  \\
   18 \\
\end{bmatrix}

\end{equation*}\\
Solving, we get:
\begin{center}
    

$

  \begin{bmatrix}
    x_1  \\
   x_2  \\

   \lambda \\
\end{bmatrix}
=\begin{bmatrix}
    10  \\
   8  \\

   4 \\
\end{bmatrix}

$
\end{center}

Note That Dual Feasibility and Complimentary Slackness are trivially true as there are no inequality constraints.
\end{frame}

%----------------------------------------------------------------------------------------

\begin{frame}
\frametitle{Question 3.10}
Solve the same minimization problem as in 3.9,
subject to the constraint $ g( \textbf{x} )=x_1+x_2-18 \geq 0 $\\
Solution:\\
\begin{equation*}
   L(\textbf{x},\mu)=(x_1-8)^2+(x_2-6)^2 + \mu(x_1+x_2-18) 
\end{equation*}

By Dual Feasibility,$\mu \geq 0$.\\
By complimentary slackness: $\mu(x_1+x_2-18)=0$\\
By Stationarity,

\begin{equation*}
   \nabla_{x} L( \textbf{x}^{opt} , \mu )=0
\end{equation*}
This gives the 2 equations $2x_1 - 16 - \mu =0$,$2x_2 - 12 - \mu=0$.
\end{frame}
%----------------------------------------------------------------------------------------
\begin{frame}
\frametitle{Question 3.10}

 2 equations $2x_1 - 16 - \mu =0$, $2x_2 - 12 - \mu=0$.\\
From primal feasibility, we get the third equation: $x_1+x_2-18=0$.
\begin{equation*}
    \begin{bmatrix}
    2       & 0 & -1  \\
   0       & 2 & -1  \\

   1      & 1 & 0 \\
\end{bmatrix}
  \begin{bmatrix}
    x_1  \\
   x_2  \\

   \mu \\
\end{bmatrix}
=\begin{bmatrix}
    16  \\
   12  \\
   18 \\
\end{bmatrix}

\end{equation*}\\
Solving, we get:
\begin{center}
    

$

  \begin{bmatrix}
    x_1  \\
   x_2  \\

   \mu \\
\end{bmatrix}
=\begin{bmatrix}
    10  \\
   8  \\
   4 \\
\end{bmatrix}

$
\end{center}

Note That Dual Feasibility is satisfied as $\mu \geq 0$\\
Note That Complimentary slackness is also satisfied as\\
\begin{center}
$\mu(x_1+x_2-18)=4(10+8-18)=0$

\end{center}


\end{frame}
%----------------------------------------------------------------------------------------

\begin{frame}
\frametitle{Question 3.11}
Generalize Lagrangian Multiplier Method for
\begin{equation*}
    \min_{x} f(\textbf{x})
\end{equation*}
Subject to the constraint $ g( \textbf{x} ) \geq 0 $\\
Solution:\\
\begin{equation*}
   L(\textbf{x},\lambda)=f(\textbf{x})-\mu g(\textbf{x})
\end{equation*}

\begin{equation*}
   \nabla_{x} f( \textbf{x}^{opt}) , \mu )=\mu \nabla_{x} g(\textbf{x}^{opt})
\end{equation*}

\begin{equation*}
    \mu \geq 0, \mu g(\textbf{x}^{opt})=0
\end{equation*}
  






\end{frame}

\end{document} 