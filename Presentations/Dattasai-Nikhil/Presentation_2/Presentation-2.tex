%%%%%%%%%%%%%%%%%%%%%%%%%%%%%%%%%%%%%%%%%
% Beamer Presentation
% LaTeX Template
% Version 1.0 (10/11/12)
%
% This template has been downloaded from:
% http://www.LaTeXTemplates.com
%
% License:
% CC BY-NC-SA 3.0 (http://creativecommons.org/licenses/by-nc-sa/3.0/)
%
%%%%%%%%%%%%%%%%%%%%%%%%%%%%%%%%%%%%%%%%%

%----------------------------------------------------------------------------------------
%	PACKAGES AND THEMES
%----------------------------------------------------------------------------------------

\documentclass{beamer}

\mode<presentation> {

% The Beamer class comes with a number of default slide themes
% which change the colors and layouts of slides. Below this is a list
% of all the themes, uncomment each in turn to see what they look like.

%\usetheme{default}
%\usetheme{AnnArbor}
%\usetheme{Antibes}
%\usetheme{Bergen}
%\usetheme{Berkeley}
%\usetheme{Berlin}
%\usetheme{Boadilla}
%\usetheme{CambridgeUS}
%\usetheme{Copenhagen}
%\usetheme{Darmstadt}
%\usetheme{Dresden}
%\usetheme{Frankfurt}
%\usetheme{Goettingen}
%\usetheme{Hannover}
%\usetheme{Ilmenau}
%\usetheme{JuanLesPins}
%\usetheme{Luebeck}
\usetheme{Madrid}
%\usetheme{Malmoe}
%\usetheme{Marburg}
%\usetheme{Montpellier}
%\usetheme{PaloAlto}
%\usetheme{Pittsburgh}
%\usetheme{Rochester}
%\usetheme{Singapore}
%\usetheme{Szeged}
%\usetheme{Warsaw}

% As well as themes, the Beamer class has a number of color themes
% for any slide theme. Uncomment each of these in turn to see how it
% changes the colors of your current slide theme.

%\usecolortheme{albatross}
%\usecolortheme{beaver}
%\usecolortheme{beetle}
%\usecolortheme{crane}
%\usecolortheme{dolphin}
%\usecolortheme{dove}
%\usecolortheme{fly}
%\usecolortheme{lily}
%\usecolortheme{orchid}
%\usecolortheme{rose}
%\usecolortheme{seagull}
%\usecolortheme{seahorse}
%\usecolortheme{whale}
%\usecolortheme{wolverine}

%\setbeamertemplate{footline} % To remove the footer line in all slides uncomment this line
%\setbeamertemplate{footline}[page number] % To replace the footer line in all slides with a simple slide count uncomment this line

%\setbeamertemplate{navigation symbols}{} % To remove the navigation symbols from the bottom of all slides uncomment this line
}

\usepackage{graphicx} % Allows including images
\usepackage{booktabs} % Allows the use of \toprule, \midrule and \bottomrule in tables

%----------------------------------------------------------------------------------------
%	TITLE PAGE
%----------------------------------------------------------------------------------------

\title[Short title]{Optimization } % The short title appears at the bottom of every slide, the full title is only on the title page

\author{MA17BTECH11002,MA17BTECH11007} % Your name
\institute[IITH] % Your institution as it will appear on the bottom of every slide, may be shorthand to save space
{
Simplex Method \\ % Your institution for the title page
\medskip
\textit{Problem 3} % Your email address
}
\date{February 28,2019} % Date, can be changed to a custom date

\begin{document}

\begin{frame}
\titlepage % Print the title page as the first slide
\end{frame}

\begin{frame}
\frametitle{Problem}

Solve the linear programming problem using the Simplex Method.\newline

Minimize $ f = -40x_{1}-100x_{2}$\newline\newline
Subject to:\newline\newline
\centering 
$10x_{1}+5x_{2} $\leq$ 2500$,\quad \quad \quad \quad \textcircled{1}\newline

$4x_{1}+10x_{2} $\leq$ 2000$ ,\quad \quad \quad \quad \textcircled{2}\newline

$2x_{1}+3x_{2} $\leq$ 900$, \quad \quad \quad \quad \quad \textcircled{3}\newline

$x_{1}$\geq$0$,
$x_{2}$\geq$0$


\end{frame}


\begin{frame}
\frametitle{Solution}
The Problem is converted to canonical form by adding slack,surplus and artificial variables as appropriate.\newline

$\Rightarrow$ As the constraint - 1 is of type '$\leq$' we should add slack variable $S_{1}$ \newline

$\Rightarrow$ As the constraint - 2 is of type '$\leq$' we should add slack variable $S_{2}$\newline

$\Rightarrow$ As the constraint - 3 is of type '$\leq$' we should add slack variable $S_{3}$\newline

\end{frame}

\begin{frame}
After introducing slack variables, we have\newline

f = $40x_{1}+ 100x_{2} +0S_{1} + 0S_{2} + 0S_{3}$\newline

subject to:
\centering
$10x_{1} + 5x_{2} + S_{1} = 2500$\newline

$4x_{1} + 10x_{2} + S_{2} = 2000$\newline

$2x_{1} + 3x_{2} + S_{3} = 900$\newline

and $x{1},x{2},S{1},S_{2},S_{3}\geq0$\newline


\end{frame}
\begin{frame}
\frametitle{Table}
\begin{table}
\begin{tabular}{l l l l l l l l l }
\toprule
\textbf{Iteration-1} & \textbf{} & \textbf{$C_{j}$} \textbf{40} & \textbf{100} & \textbf{0} & \textbf{0} & \textbf{0} & \textbf{       }\\

\midrule
B & $X_{B}$ & $x_{1}$ & $x_{2}$ & $S_{1}$ & $S_{2}$ & $S_{3}$ & Min Ratio=$\frac{X_{B}}{x_{2}}$\\
$S_{1}$ & $2500$ & 10 & 5 & 1 & 0 & 0 & $\frac{2500}{5}$=500\\

$S_{2}$ & $2000$ & 4 & (10) & 0 & 1 & 0 & $\frac{2000}{10}$=200\\

$S_{3}$ & $900$ & 2 & 3 & 0 & 0 & 1 & $\frac{900}{3}$=300\\

Z = 0 & $Z_{j}$  & 0 & 0 & 0 & 0 & 0 &  \\
\bottomrule

   & $Z_{j}-C_{j}$ & -40 & -100 & 0 & 0 & 0 &  \\
\end{tabular}

\end{table}\newline

The minimum $Z_{j}-C_{j}$ is -100 and its column index is 2.So, the entering variable is $x_{2}$.\newline
The minimum ratio is 200 and its row index is 2. So, the leaving basis variable is $s_{2}$.\newline
$\therefore$ The pivot element is 10.

\end{frame}

\begin{frame}

Entering = $x_{2}$, Departing = $S_{2}$ , Key Element = 10.\newline

$R_{2}(new)$ = $R_{2}(old)/10$\newline

$R_{1}(new)$ = $R_{1}(old)$ - $5R_{1}(new)$\newline

$R_{3}(new)$ = $R_{3}(old)$ - $3R_{2}(new)$\newline


    
\end{frame}




\begin{frame}
\frametitle{Table}
\begin{table}
\begin{tabular}{l l l l l l l l l }
\toprule
\textbf{Iteration-2} & \textbf{} & \textbf{$C_{j}$} \textbf{40} & \textbf{100} & \textbf{0} & \textbf{0} & \textbf{0} & \textbf{       }\\

\midrule
B & $X_{B}$ & $x_{1}$ & $x_{2}$ & $S_{1}$ & $S_{2}$ & $S_{3}$ & Min Ratio=$\frac{X_{B}}{x_{2}}$\\
$S_{1}$ & $1500$ & 8 & 0 & 1 & $-\frac{1}{2}$ & 0 & \\

$x_{2}$ & $200$ & $\frac{2}{5}$ & 1 & 0 & $\frac{1}{10}$ & 0 & \\

$S_{3}$ & 300 & $\frac{4}{5}$ & 0 & 0 & $-\frac{3}{10}$ & 1 & \\

Z = 20000 & $Z_{j}$  & 40 & 100 & 0 & 10 & 0 &  \\
\bottomrule

   & $Z_{j}-C_{j}$ & 0 & 0 & 0 & 10 & 0 &  \\
\end{tabular}

\end{table}\newline


\end{frame}

\begin{frame}

Since all $Z_{j}-C_{j} \geq 0$\newline

Hence optimal solution arrived with value of variables as:\newline

$x_{1} = 0,x_{2} = 200$\newline

Max Z = 20000\newline

Min -Z = -20000\newline

Hence, the minimum of  $ f = -40x_{1}-100x_{2}$ subject to the given conditions is -20000
    
\end{frame}


\begin{frame}[fragile] % Need to use the fragile option when verbatim is used in the slide
\frametitle{Python Code}
\begin{example}
\begin{verbatim}
   from scipy.optimize import linprog

C = [-40,-100]      #cost function
A = [[10,5],[4,10],[2,3]]   #Constraint matrix
B = [2500,2000,900]     #RHS of constraints
x0_bounds = (0,None)    #making sure x1 and x2 are >=0
x1_bounds = (0,None)
#call the lin prog function from the library we imported
res = linprog(C,A_ub=A,b_ub=B,bounds=(x0_bounds,x1_bounds)
,options={"disp":True})
   
\end{verbatim}
\end{example}
\end{frame}



\begin{frame}
\Huge{\centerline{The End}}
\quad \quad\quad \quad \quad Thank You!
\end{frame}





%----------------------------------------------------------------------------------------

\end{document}