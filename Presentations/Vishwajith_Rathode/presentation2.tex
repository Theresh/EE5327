%%%%%%%%%%%%%%%%%%%%%%%%%%%%%%%%%%%%%%%%%
% Beamer Presentation
% LaTeX Template
% Version 1.0 (10/11/12)
%
% This template has been downloaded from:
% http://www.LaTeXTemplates.com
%
% License:
% CC BY-NC-SA 3.0 (http://creativecommons.org/licenses/by-nc-sa/3.0/)
%
%%%%%%%%%%%%%%%%%%%%%%%%%%%%%%%%%%%%%%%%%

%----------------------------------------------------------------------------------------
%	PACKAGES AND THEMES
%----------------------------------------------------------------------------------------

\documentclass{beamer}

\mode<presentation> {

% The Beamer class comes with a number of default slide themes
% which change the colors and layouts of slides. Below this is a list
% of all the themes, uncomment each in turn to see what they look like.

%\usetheme{default}
%\usetheme{AnnArbor}
%\usetheme{Antibes}
%\usetheme{Bergen}
%\usetheme{Berkeley}
%\usetheme{Berlin}
%\usetheme{Boadilla}
%\usetheme{CambridgeUS}
%\usetheme{Copenhagen}
%\usetheme{Darmstadt}
%\usetheme{Dresden}
%\usetheme{Frankfurt}
%\usetheme{Goettingen}
%\usetheme{Hannover}
%\usetheme{Ilmenau}
%\usetheme{JuanLesPins}
%\usetheme{Luebeck}
\usetheme{Madrid}
%\usetheme{Malmoe}
%\usetheme{Marburg}
%\usetheme{Montpellier}
%\usetheme{PaloAlto}
%\usetheme{Pittsburgh}
%\usetheme{Rochester}
%\usetheme{Singapore}
%\usetheme{Szeged}
%\usetheme{Warsaw}

% As well as themes, the Beamer class has a number of color themes
% for any slide theme. Uncomment each of these in turn to see how it
% changes the colors of your current slide theme.

%\usecolortheme{albatross}
%\usecolortheme{beaver}
%\usecolortheme{beetle}
%\usecolortheme{crane}
%\usecolortheme{dolphin}
%\usecolortheme{dove}
%\usecolortheme{fly}
%\usecolortheme{lily}
%\usecolortheme{orchid}
%\usecolortheme{rose}
%\usecolortheme{seagull}
%\usecolortheme{seahorse}
%\usecolortheme{whale}
%\usecolortheme{wolverine}

%\setbeamertemplate{footline} % To remove the footer line in all slides uncomment this line
%\setbeamertemplate{footline}[page number] % To replace the footer line in all slides with a simple slide count uncomment this line

%\setbeamertemplate{navigation symbols}{} % To remove the navigation symbols from the bottom of all slides uncomment this line
}

\usepackage{graphicx} % Allows including images
\usepackage{booktabs} % Allows the use of \toprule, \midrule and \bottomrule in tables

%----------------------------------------------------------------------------------------
%	TITLE PAGE
%----------------------------------------------------------------------------------------

\title[]{Problems 43} % The short title appears at the bottom of every slide, the full title is only on the title page

\author{Jatoth Vishwajith Rathod} % Your name
\institute[] % Your institution as it will appear on the bottom of every slide, may be shorthand to save space
{
IIT Hyderabad \\ % Your institution for the title page
\medskip
\textit{ee16btech11014@iith.ac.in} % Your email address 
}
\date{\today} % Date, can be changed to a custom date

\begin{document}

\begin{frame}
\titlepage % Print the title page as the first slide
\end{frame}



%----------------------------------------------------------------------------------------
%	PRESENTATION SLIDES
%----------------------------------------------------------------------------------------

%------------------------------------------------


\begin{frame}
\frametitle{Question 43}

\begin{problem}
The following table shows the information on the availability of supply to each warehouse, the requirement of each market and unit of transportation cost (in rupees) from each warehouse to each market.
%\medskip
\begin{table}[!h]
\centering
\begin{tabular}{c c c c c c c}
& & Market & & & & \\
& & $M_1$ & $M_2$ & $M_3$ & $M_4$ & Supply \\
& $W_1$ & 6 & 3 & 5 & 4 & 22 \\
Warehouse & $W_2$ & 5 & 9 & 2 & 7 & 15 \\
& $W_3$ & 5 & 7 & 8 & 6 & 8 \\
Requirement & & 7 & 12 & 17 & 9 & 
\end{tabular}
\end{table}
\medskip
The present transportation schedule is as follows: \\
$W_1$ to $M_2$: 12 units; $W_1$ to $M_3$: 1 unit; $W_1$ to $M_4$: 9 units; $W_2$ to $M_3$: 15 units; $W_3$ to $M_1$: 7 units and $W_3$ to $M_3$: 1 unit. Then the minimum total transportation cost (in rupees) using MODI method is 

%\begin{enumerate}[(A)]
%\begin{multicols}{4}
%\setlength\itemsep{1em}
%\item 150
%\item 149
%\item 148
%\item 147
%\end{multicols}
%\end{enumerate}
\end{problem}

\end{frame}




%------------------------------------------------

\begin{frame}
\frametitle{Solution}
\begin{problem} 
Find the optimal solution for the given transportation problem
\begin{table}[!h]
\begin{center}
\begin{tabular}{l  l | l l l l | l  }
                 
                 & S/D & 1 & 2 & 3 & 4 &  SUPPLY  \\
\hline
& 1 & 6 & 3 & 5 & 4 & 22 \\ 
& 2 & 5 & 9 & 2 & 7 & 15  \\ 
& 3 & 5 & 7 & 8 & 6 & 8 \\ 
\hline
&DEMAND& 7&12&17&9\\ 
\end{tabular}
\end{center}
\caption{12}
\end{table}
\end{problem}
\end{frame}

%------------------------------------------------
%------------------------------------------------

\begin{frame}
\frametitle{Solution}
\item 
Make the objective function for given costs
\\

$
f=6x_{11}+3x_{12}+5x_{13}+4x_{14}+
5x_{21}+9x_{22}+2x_{23}+7x_{24}+
5x_{31}+7x_{32}+8x_{33}+6x_{34} 
$
\\
\item 
\end{frame}

%------------------------------------------------



%------------------------------------------------

\begin{frame}{Solution}
Make the constraints
\\
$
6x_{11}+3x_{12}+5x_{13}+4x_{14} \leq 22$ \\$
5x_{21}+9x_{22}+2x_{23}+7x_{24} \leq 15$ \\$
5x_{31}+7x_{32}+8x_{33}+6x_{34} \leq 8$ \\$
-6x_{11}-5x_{21}-5x_{31} \leq -7  $ \\$
-3x_{12}-9x_{22}-7x_{32} \leq -12 $ \\$
-5x_{13}-2x_{23}-8x_{33} \leq -17 $ \\$
-4x_{14}-7x_{24}-6x_{34} \leq -9 $ \\$
x_{11},x_{12},x_{13},x_{14},x_{21},x_{22},x_{23},x_{24},
x_{31},x_{32},x_{33},x_{34} \geq 0
$
\end{frame}




\begin{frame}{Solution}
from cvxopt import matrix
from cvxopt import solvers\\
A = matrix([[1.0 ,1.0 ,1.0 ,1.0 ,0.0 ,0.0 ,
0.0 ,0.0 ,0.0 ,0.0 ,0.0 ,0.0] ,
[0.0 ,0.0 ,0.0 ,0.0 ,1.0 ,1.0 ,
1.0 ,1.0 ,0.0 ,0.0 ,0.0 ,0.0 ] ,
[0.0 ,0.0 ,0.0 ,0.0 ,0.0 ,0.0 ,
0.0 ,0.0 ,1.0 ,1.0 ,1.0 ,1.0] ,
[ - 1.0 ,0.0 ,0.0 ,0.0 , - 1.0 ,0.0 ,
0.0 ,0.0 , - 1.0 ,0.0 ,0.0 ,0.0] ,
[0.0 , - 1.0 ,0.0 ,0.0 ,0.0 , - 1.0 ,
0.0 ,0.0 ,0.0 , - 1.0 ,0.0 ,0.0] ,
[0.0 ,0.0 , - 1.0 ,0.0 ,0.0 ,0.0 ,
- 1.0 ,0.0 ,0.0 ,0.0 , - 1.0 ,0.0] ,
[0.0 ,0.0 ,0.0 , - 1.0 ,0.0 ,0.0 ,
0.0 , - 1.0 ,0.0 ,0.0 ,0.0 , - 1.0] ,
[ - 1.0 ,0.0 ,0.0 ,0.0 ,0.0 ,0.0 ,
0.0 ,0.0 ,0.0 ,0.0 ,0.0 ,0.0] ,
[0.0 , - 1.0 ,0.0 ,0.0 ,0.0 ,0.0 ,
0.0 ,0.0 ,0.0 ,0.0 ,0.0 ,0.0] ,
[0.0 ,0.0 , - 1.0 ,0.0 ,0.0 ,0.0 ,
0.0 ,0.0 ,0.0 ,0.0 ,0.0 ,0.0] ,
[0.0 ,0.0 ,0.0 , - 1.0 ,0.0 ,0.0 ,
0.0 ,0.0 ,0.0 ,0.0 ,0.0 ,0.0] ,
[0.0 ,0.0 ,0.0 ,0.0 , - 1.0 ,0.0 ,
0.0 ,0.0 ,0.0 ,0.0 ,0.0 ,0.0] ,
[0.0 ,0.0 ,0.0 ,0.0 ,0.0 , - 1.0 ,
0.0 ,0.0 ,0.0 ,0.0 ,0.0 ,0.0] ,
[0.0 ,0.0 ,0.0 ,0.0 ,0.0 ,0.0 ,
- 1.0 ,0.0 ,0.0 ,0.0 ,0.0 ,0.0] ,
[0.0 ,0.0 ,0.0 ,0.0 ,0.0 ,0.0 ,
0.0 , - 1.0 ,0.0 ,0.0 ,0.0 ,0.0] ,
[0.0 ,0.0 ,0.0 ,0.0 ,0.0 ,0.0 ,
0.0 ,0.0 , - 1.0 ,0.0 ,0.0 ,0.0] ,
[0.0 ,0.0 ,0.0 ,0.0 ,0.0 ,0.0 ,
0.0 ,0.0 ,0.0 , - 1.0 ,0.0 ,0.0] ,
[0.0 ,0.0 ,0.0 ,0.0 ,0.0 ,0.0 ,
0.0 ,0.0 ,0.0 ,0.0 , - 1.0 ,0.0] ,
[0.0 ,0.0 ,0.0 ,0.0 ,0.0 ,0.0 ,
0.0 ,0.0 ,0.0 ,0.0 ,0.0 , - 1.0]])

\end{frame}


\begin{frame}{Solution}
b = matrix ([22.0 , 15.0 , 8.0 ,
- 7.0 , - 12.0 , - 17.0 ,
- 9.0 ,
0 ,0 ,0 ,0 ,0 ,0 ,0 ,0 ,0 ,0 ,0 ,0])\\
c= matrix ( [ 6.0 ,3.0 ,5.0 ,4.0 ,
5.0 , 9.0 , 2.0 , 7.0 , 5.0 , 7.0 ,8.0 , 6.0])\\
sol = solvers.sdp ( c , A.T , b )

print ( sol ['x'] )

print(sol['x'][0]*6.0 + sol['x'][1]*3.0 + sol['x'][2]*5.0 + sol['x'][3]*4.0 + sol['x'][4]*5.0 + sol['x'][5]*9.0 + sol['x'][6]*2.0 +sol['x'][7]*7.0 +sol['x'][8]*5.0 + sol['x'][9]*7.0 + sol['x'][10]*8.0 + sol['x'][11]*6.0)
\end{frame}
\begin{frame}{Solution}

\begin{figure}[!ht]
\centering
\includegraphics[width=\columnwidth]{./figs/2.15.eps}
\caption{ minimal total transportation cost is}
\label{fig.2.15}	
\end{figure}
%

\end{frame}


\begin{frame}{Solution}
\begin{center}
    \includegraphics[scale=0.5]{answer1.png}
\end{center}
\end{frame}
\begin{frame}{And last}
\begin{center}
    Thank You
\end{center}
\end{frame}
\end{document} 